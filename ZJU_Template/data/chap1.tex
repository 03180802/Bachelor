\chapter{绪论}

\section{机器学习与正则化}

机器学习是人工智能的一个子领域,它着眼于设计和开发能自动从数据中生成出模型的算法。机器学习又
被细分为监督学习、无监督学习以及半监督学习、增强学习等类别,但是它们的思路都是相似的,亦即对
问题进行建模,并寻找一个最适合给定约束条件的函数或模型,通常约束条件直接来自于训练数据。如果
训练数据的数量足够多,我们通常能得到比较好的结果,然而大多数时候训练数据总是不够的,此时我们
通常会遇到有很多(甚至无穷多个)满足约束条件的解的情况,并且大部分这样的解在未知数据上的表现
非常差。换句话说,这样的问题是数值不稳定的,或者说泛化性能很差。

一种解决办法就是正则化(Regularization),亦即对可能解空间进行限制,这种技术最初由
Tikhonov 和 Arsenin \cite{tikhomirov1960dsf} 提出,用于解决矩阵求逆的问题,并且后
来被成功地应用到机器学习中来。许多机器学习的算法(例如支持向量机)都可以看成是某
种形式的正则化。简单来说,正则化可以看作是关于待解决问题的某种先验的知识,例如,在
脊回归(Ridge Reguression)中的正则化可以看作是将参数的先验分布定为高斯分布的结果,
而流形正则化(Manifold Regularization)\cite{On-Manifold-Regularization} 则是编码了
数据点都分布在一个低维流形上这样一个先验假设。

虽然正则化如今已经成为机器学习中广泛使用的一种技术,大多数情况人们都只考虑了问题在空间上的
属性。当我们遇到随时间变化的数据(例如,动态网页、博客内容或者股票价格等)时,一个很自然的
想法就是要保证数据在时间上的平滑性,这样的假设通常能帮助我们得到更健壮的解。

\section{写点什么吧}

恩,反正这一章就是绪论。
